\documentclass[UTF8]{ctexart}

\author{Tony\_Wong}
\date{August 17, 2019}
\title{\huge \texttt{NOIP}知识点}
\usepackage{listings}
\usepackage{fontspec}
\usepackage[colorlinks,linkcolor=blue]{hyperref}

\lstset{
	basicstyle=\fontspec{Monaco for Powerline}
}

\begin{document}

%\maketitle

%\tableofcontents

\section{前言}
本文记录了$NOIP$涉及知识点和部分代码实现

\section{基础算法}
主要有\textbf{模拟}、\textbf{枚举}、\textbf{贪心}、\textbf{二分}、\textbf{分治}、倍增\\
\textbf{加粗}为重点知识点

\subsection{模拟、枚举、贪心、分治、倍增}
遇到具体题目,具体分析,\emph{灵活运用}即可

\subsection{二分}
\subsubsection{二分答案}
一般遇到题目里出现\emph{最大值最小,最小值最大}这样的字眼,一般都需要使用\textbf{二分答案}\\

\begin{quote}
	在可能的答案区间里找出问题的答案,大多数情况下用于求解满足某种条件下的最大(小)值,前提是答案具有单调性
\end{quote}

模板题目:$Luogu\ P1873$砍树

主体代码:\\
\begin{lstlisting}
while (l <= r) {
    mid = (l + r) >> 1;
    if (check(mid)) {
        ans = mid;
        l = mid + 1;
    } else {
        r = mid - 1;
    }
}
\end{lstlisting}

具体\texttt{check()}函数部分根据具体问题编写。而且\texttt{l = ...和r = ...}的部分也要根据具体情况灵活更改

\subsubsection{二分查找}
经常使用\texttt{C++\ STL}内的\texttt{lower\_bound()和upper\_bound()}来完成,复杂度$O(\log n)$

\begin{itemize}
	\item \texttt{lower\_bound(begin, end, num)}:从数组的begin位置到end-1位置二分查找第一个大于或等于num的数字,找到返回该数字的地址,不存在则返回end,通过返回的地址减去起始地址begin,得到找到数字在数组中的下标
	\item \texttt{upper\_bound(begin, end, num)}:从数组的begin位置到end-1位置二分查找第一个大于num的数字,找到返回该数字的地址,不存在则返回end,通过返回的地址减去起始地址begin,得到找到数字在数组中的下标
	\item \texttt{lower\_bound(begin, end, num, greater<type>())}:第一个小于或等于num
	\item \texttt{upper\_bound(begin, end, num, greater<type>())}:第一个小于num
\end{itemize}

实例:
\begin{lstlisting}
int num[6] = {1, 2, 4, 7, 15, 34}; 
int pos1 = lower_bound(num, num + 6, 7) - num;
//pos1 = 3   num[pos1] = 7
int pos2 = upper_bound(num, num + 6, 7) - num;
//pos2 = 4   num[pos2] = 15
\end{lstlisting}

\subsection{离散化}

假设\texttt{int}范围内有\texttt{n}个整数\texttt{a[1]\textasciitilde a[n]},去重后有\texttt{m}个整数;则可以把\texttt{a[i]}用\texttt{1\textasciitilde m}间的整数来代替,并保持大小顺序不变。实现了用小整数代替大整数运行算法

\begin{lstlisting}
void discrete() {
    sort(a + 1, a + 1 + n);
    for (int i = 1; i <= n; ++i) {
        if (i == 1 || a[i] != a[i - 1]) {
            b[++m] = a[i];
        }
    }
}
int query(int x) {
    return lower_bound(b + 1, b + 1 + m, x) - b;
}
\end{lstlisting}

\texttt{i}代替的数值为\texttt{b[i]},\texttt{a[i]}被\texttt{query(a[i])}代替

\section{动态规划}
\subsection{线性DP、高维DP、背包DP}

根据具体问题具体分析,拥有动态规划思想即可

\subsubsection{LIS问题}



\subsection{树形DP}

以图的方式建树,之后按照递归的写法,从根节点初始化到叶节点,再一层一层向上根据状态转移方程更新

\begin{lstlisting}
void DP(int x) {
    该节点初始化
    for (int i = head[x]; i; i = e[i].next) {
        int y = e[i].to;
        DP(y);
        根据状态转移方程更新
    }
}
\end{lstlisting}

模板题目:$Luogu\ P1352$没有上司的舞会\\
还有局域网\texttt{192.168.0.123}里山东1127、1130题

\subsection{区间DP}
\begin{quote}
	区间dp就是在区间上进行动态规划,求解一段区间上的最优解。主要是通过合并小区间的 最优解进而得出整个大区间上最优解的dp算法。
\end{quote}

区间长度作为\textbf{阶段},左端点作为\textbf{状态},转移方程作为\textbf{决策}

重点:\textbf{分清阶段、状态、决策}\\

经典例题:$Luogu\ P1880$石子合并

\begin{lstlisting}
for (int len = 2; len <= n; ++len) { // 阶段
    for (int l = 1; l <= n - len + 1; ++l) { // 状态:左端点
        int r = l + len - 1; // 状态:右端点
        for (int k = l; k < r; ++k) { // 决策
            f[l][r] = min(f[l][r], f[l][k] + f[k + 1][r]);
        }
        f[l][r] += sum[r] - sum[l - 1];
    }
}
\end{lstlisting}

\subsection{状压DP}
即使用二进制状态压缩来优化DP

没学过,更没写过。逃...

但我知道例题:$Luogu\ P1879$玉米田

\section{数论}

\subsection{欧几里得算法}

\begin{lstlisting}
LL gcd(LL a, LL b) {
    return b == 0 ? a : gcd(b, a % b);
}
\end{lstlisting}

\subsection{埃式筛法}

\begin{lstlisting}
bool vis[maxn];
int prime[maxn];
void getprime(int n) {
    int m = (int)sqrt(n + 0.5), num = 0;
    memset(vis, 0, sizeof(vis));
    vis[0] = vis[1] = 1;
    for (int i = 2; i <= m; ++i) if (!vis[i]) {
        prime[++num] = i;
        for (int j = i * i; j <= n; j += i) vis[j] = 1;
    }
}
\end{lstlisting}

\subsection{扩展欧几里得(线性同余方程)}

\begin{lstlisting}
void exgcd(LL a, LL b, LL& d, LL& x, LL& y) {
    if (!b) { d = a; x = 1; y = 0; }
    else { exgcd(b, a % b, d, y, x); y -= x * (a / b); }
}
\end{lstlisting}

\subsection{乘法逆元}

在大数相除取余时要转换为乘以乘法逆元,求逆元方法:

\begin{lstlisting}
LL inv(LL a, LL n) {
    LL d, x, y;
    exgcd(a, n, d, x, y);
    return d == 1 ? (x + n) % n : -1;
}
\end{lstlisting}

\subsection{快速乘,快速幂}

\begin{lstlisting}
LL mul_mod(LL a, LL b, LL n){
    LL res = 0;
    while (b > 0) {
        if (b & 1) res = (res + a) % n;
        a = (a + a) % n;
        b >>= 1;
    }
    return res;
}

LL pow_mod(LL a, LL p, LL n){
    int res = 1;
    while (p) {
        if (p & 1) res = 1LL * res * a % n;
        a = 1LL * a * a % n;
        p >>= 1;
    }
    return res;
}
\end{lstlisting}

\section{数据结构}

\subsection{单调栈,单调队列}

即在普通的栈、队列基础上维护整个栈和队列的单调性

例题:$Luogu\ P3467$PLA-Postering

\subsection{堆}

不多说,可以使用\texttt{priority\_queue}

\subsection{并查集}

\textbf{动态维护}若干个不重叠的集合,并支持\textbf{合并与查询}

\begin{lstlisting}
int ufs[maxn];
for (int i = 1; i <= n; ++i) ufs[i] = i;
int find(int x) {
    return x == ufs[x] ? x : ufs[x] = find(ufs[x]);
}
void unionn(int x, int y) {
    ufs[get(x)] = get(y);
}
\end{lstlisting}

带权并查集:

\begin{lstlisting}
int find(int x) {
    if (ufs[x] == x) return x;
    int fx = find(ufs[x]);
    d[x] += d[ufs[x]];
    return ufs[x] = fx;
}
void union(int x, int y) {
    int fx = find(x), fy = find(y);
    ufs[fx] = fy; d[fx] = size[fy];
    size[fy] += size[fx];
}
\end{lstlisting}

带撤销并查集:

\begin{lstlisting}
int ufs[maxn];
void init() {
    for (int i = 0; i <= n; ++i) {
        ufs[i] = i;
    }
}
inline void sroot(int u) { //换u为根
    for (int i = 0, fa = ufs[u]; u; fa = ufs[u]) {
        ufs[u] = i; i = u; u = fa;
    }
}
void connect(int u, int v) {
    ufs[u] = v;
}
void deleteuv(int u, int v) {
    ufs[v] = 0;
}
void query(int u, int v) {
    for (; v != u && v; v = ufs[v]);
    puts(v == u ? "Yes" : "No");
}
\end{lstlisting}

具体讲解,例题在我的博客\url{https://www.luogu.org/blog/tony180116/DataStructure-UFS}

\subsection{树状数组}

维护前缀和的数据结构,支持\textbf{单点增加,查询前缀和}

\texttt{lowbit(n)}操作:非负整数$n$在二进制表示下最低位的1及其后边所有的0构成的数值\\

对于一个原序列\texttt{a[]},可以建立一个数组\texttt{tree[]},来保存\texttt{a}的区间\texttt{[x-lowbit(x)+1, x]}内值的和,即

\[\mathtt{tree[x] = \sum_{i=x-lowbit(x)+1}^x{a[i]}}\]

同时$tree$数组可以看成一个树形结构,并满足以下性质

\begin{itemize}
	\item 每个节点\texttt{tree[x]}保存以$x$为根的子树中所有叶节点的和
	\item 每个节点\texttt{tree[x]}的子节点个数等于\texttt{lowbit(x)}的位数
	\item 除树根外,每个节点\texttt{tree[x]}的父亲节点为\texttt{tree[x+lowbit(x)]}
	\item 树的深度为$\log_2n$ 
\end{itemize}

\begin{lstlisting}
int tree[maxn], n, m;
int lowbit(int k) { return k & -k; }
void add(int x, int k) {
    for (; x <= maxn; x += x & -x) tree[x] += y;
}
int query(int x) {
    int ans = 0;
    for (; x; x -= x & -x) ans += tree[x];
    return ans;
}
int init() {
    for (int i = 1; i <= n; ++i) {
        add(i, a[i]);
    }
}
\end{lstlisting}

模板:$Luogu\ P3374$树状数组1\ -\ 单点修改,区间查询\\
$Luogu\ P3368$树状数组2\ -\ 区间修改,单点查询\\
$Luogu\ P3372$线段树1\ -\ 区间修改,区间查询

\subsection{线段树}

基于分治思想的二叉树形数据结构,可以用于\textbf{区间}上的数据维护

\begin{lstlisting}
struct SegmentTreeNode { //树上的节点
    int l, r;                   //节点表示区间的左右端点
    long long sum, add;         //区间和和add的延迟标记
    #define l(x) tree[x].l      //方便访问
    #define r(x) tree[x].r
    #define sum(x) tree[x].sum
    #define add(x) tree[x].add
} tree[maxn << 2];
int a[maxn], n, m;
void pushup(int p) {
    sum(p) = sum(p<<1) + sum(p<<1|1);
}
void build(int p, int l, int r) { //建树
    l(p) = l, r(p) = r;           //设置左右端点
    if (l == r) { sum(p) = a[l]; return; } //到达叶子节点
    int mid = (l + r) >> 1;
    build(p<<1, l, mid);         //递归构建左右树
    build(p<<1|1, mid + 1, r);
    pushup(p);
}
void pushdown(int p) { //下传延迟标记
    if (add(p)) { //如果有标记
        sum(p<<1) += add(p) * (r(p<<1) - l(p<<1) + 1); //sum传至左儿子
        sum(p<<1|1) += add(p) * (r(p<<1|1) - l(p<<1|1) + 1); //右儿子
        add(p<<1) += add(p);      //延迟标记传至左儿子
        add(p<<1|1) += add(p);  //右儿子
        add(p) = 0; //本节点延迟标记清零
    }
}
void update(int p, int l, int r, int d) { //更新区间内值
    if (l <= l(p) && r >= r(p)) { //完全覆盖
        sum(p) += (long long)d * (r(p) - l(p) + 1); //更新节点信息
        add(p) += d; //打上延迟标记
        return;
    }
    pushdown(p); //下传标记
    int mid = (l(p) + r(p)) >> 1;
    if (l <= mid) update(p<<1, l, r, d);      //递归更新左右
    if (r >  mid) update(p<<1|1, l, r, d);
    pushup(p);
}
long long query(int p, int l, int r) { //查询操作
    if (l <= l(p) && r >= r(p)) return sum(p); //完全覆盖
    pushdown(p); //下传标记
    int mid = (l(p) + r(p)) >> 1;
    long long ans = 0;
    if (l <= mid) ans += query(p<<1, l, r);     //加上左右部分值
    if (r >  mid) ans += query(p<<1|1, l, r);
    return ans;
}
\end{lstlisting}

模板:$Luogu\ P3372$线段树1\ -\ 区间加,区间查询和\\
推荐题目:$SPOJ\ GSS1\~7$\ Can you answer these queries

\url{https://www.luogu.org/blog/tony180116/DataStructure-SegmentTree}

\subsection{ST表}

求静态区间最大值,查询复杂度$O(1)$

模板题目:$Luogu\ P3865$ST表

\begin{lstlisting}
Log[0] = -1;
for (int i = 1; i <= n; ++i) {
    f[i][0] = a[i];
    Log[i] = Log[i >> 1] + 1;
}
for (int j = 1; j <= LogN; ++j) {
    for (int i = 1; i + (1 << j) - 1 <= n; ++i)
        f[i][j] = max(f[i][j - 1], f[i + (1 << j - 1)][j - 1]);
}
while (m--) {
    scanf("%d %d", &l, &r);
    int s = Log[r - l + 1];
    printf("%d\n", max(f[l][s], f[r - (1 << s) + 1][s]));
}
\end{lstlisting}

\subsection{哈希表}

模板题目:$Luogu\ P4305$不重复数字

\begin{lstlisting}
#define p 100003
#define hash(a) a%p
int h[p], t, n, x;
int find(int x) {
    int y;
    if (x < 0) y = hash(-x);
    else y = hash(x);
    while (h[y] && h[y] != x) y = hash(++y);
    return y;
}
void push(int x) {
    h[find(x)] = x;
}
bool check(int x) {
    return h[find(x)] == x;
}
\end{lstlisting}

\subsection{Trie字典树}

没见过几道题

\begin{lstlisting}
struct Trie {
    int ch[maxn][26], sz, val[maxn];
    Trie() {
        sz = 1;
        memset(ch[0], 0, sizeof(ch[0]));
        memset( val , 0, sizeof( val ));
    }
    int idx(char c) { return c - 'a'; }
    void insert(char *s, int v) {
        int u = 0, n = strlen(s);
        for (int i = 0; i < n; ++i) {
            int c = idx(s[i]);
            if (!ch[u][c]) {
                memset(ch[sz], 0, sizeof(ch[sz]));
                val[sz] = 0;
                ch[u][c] = sz++;
            }
            u = ch[u][c];
        }
        val[u] = v;
    }
    int search(char *s) {
        int u = 0, n = strlen(s);
        for (int i = 0; i < n; ++i) {
            int c = idx(s[i]);
            if (!ch[u][c]) return -1;
            u = ch[u][c];
        }
        return val[u];
    }
};
\end{lstlisting}

\section{图论}

由于我习惯用\texttt{vector<>}建图,所以这里模板均是vector写的

\subsection{最短路(Dijkstra+Heap, SPFA, Floyd)}

团队博客里有两篇相关文章

\subsubsection{Dijkstra堆优化}

不能含负权边

\begin{lstlisting}
struct Edge {
    int from, to, dis;
    Edge(int u, int v, int w): from(u), to(v), dis(w) {}
};
vector<Edge> edges;
vector<int> G[maxn];
void add(int u, int v, int w) { //单向边
    edges.push_back(Edge(u, v, w));
    int mm = edges.size();
    G[u].push_back(mm - 1);
}

struct heap {
    int u, d;
    bool operator < (const heap& a) const {
        return d > a.d;
    }
};

int n, m, d[maxn];

void Dijkstra(int s) {
    priority_queue<heap> q;
    memset(d, 0x3f, sizeof(d));
    d[s] = 0;
    q.push((heap){s, 0});
    while (!q.empty()) {
        heap top = q.top(); q.pop();
        int u = top.u, td = top.d;
        if (td != d[u]) continue;
        for (int i = 0; i < G[u].size(); ++i) {
            Edge& e = edges[G[u][i]];
            if (d[e.to] > d[u] + e.dis) {
                d[e.to] = d[u] + e.dis;
                q.push((heap){e.to, d[e.to]});
            }
        }
    }
}
\end{lstlisting}

\subsubsection{SPFA及负环}

反正比堆优化的\texttt{Dijkstra}慢就是了

\begin{lstlisting}
struct Edge {
    int from, to, val;
    Edge(int u, int v, int w): from(u), to(v), val(w) {}
};
vector<Edge> edges;
vector<int> G[maxn];
void add(int u, int v, int w) {
    edges.push_back(Edge(u, v, w));
    int mm = edges.size();
    G[u].push_back(mm - 1);
}

int dist[maxn], vis[maxn];
int n, m, w;

void SPFA(int s) {
    queue<int> q;
    q.push(s);
    dist[s] = 0; vis[s] = 1;
    while (!q.empty()) {
        int u = q.front(); q.pop();
        vis[u] = 0;
        for (int i = 0; i < G[u].size(); ++i) {
            Edge& e = edges[G[u][i]];
            if (dist[e.to] > dist[u] + e.val) {
                dist[e.to] = dist[u] + e.val;
                if (!vis[e.to]) {
                    vis[e.to] = 1;
                    q.push(e.to);
                }
            }
        }
    }
}
\end{lstlisting}

如果任意一条边被修改大于n次(执行n次松弛操作),这个图内一定存在至少一个负环

\subsection{最小生成树Kruskal}

\begin{lstlisting}
struct Edge {
    int from, to, val;
    Edge(int u, int v, int w): from(u), to(v), val(w) {}
};
vector<Edge> edges;
vector<int> G[maxn];
void add(int u, int v, int w) {
    edges.push_back(Edge(u, v, w));
    int mm = edges.size();
    G[u].push_back(mm - 1);
}
bool cmp(Edge a, Edge b) {
    if (a.val != b.val) return a.val < b.val;
    if (a.from != b.from) return a.from < b.from;
    return a.to < b.to;
}

int ufs[maxn], n, m, ans = 0, cnt = 1;
int find(int x) { return ufs[x] == x ? x : ufs[x] = find(ufs[x]); }

void Kruskal() {
    for (int i = 1; i <= n; ++i) ufs[i] = i;
    sort(edges.begin(), edges.end(), cmp);
    for (int i = 0; i < m; ++i) {
        Edge& e = edges[i];
        int x = find(e.from), y = find(e.to);
        if (x != y) {
            ufs[x] = y;
            ans += e.val;
            cnt++; //运行后若cnt!=n,则图不连通
        }
    }
}
\end{lstlisting}

\subsection{二分图匹配}

不是重点,但是学过

\subsection{Tarjan}

没学过,目前不会,溜了溜了

\subsection{树相关}

\subsubsection{树的DFS序}

在进行深度优先遍历时,依次经过的节点序列,长度为\texttt{2n},可以把树转化成区间问题

\begin{lstlisting}
void dfs(int x) {
    a[++cnt] = x;
    vis[x] = true;
    for (int i = 0; i < son[x].size(); ++i) {
        int y = son[x][i];
        if (vis[y]) continue;
        dfs(y);
    }
    a[++cnt] = x;
}
\end{lstlisting}

\subsubsection{树的深度}

\begin{lstlisting}
void dfs(int x) {
    vis[x] = true;
    for (int i = 0; i < son[x].size(); ++i) {
        int y = son[x][i];
        if (vis[y]) continue;
        dep[y] = dep[x] + 1;
        dfs(y);
    }
}
\end{lstlisting}

\subsubsection{树的重心}

\begin{lstlisting}
void dfs(int x) {
    vis[x] = true; size[x] = 1;
    int max_part = 0;
    for (int i = 0; i < son[x].size(); ++i) {
        int y = son[x][i];
        if (vis[y]) continue;
        dfs(y);
        size[x] += size[y];
        max_part = max(max_part, size[y]);
    }
    max_part = max(max_part, n - size[x]);
    if (max_part < ans) {
        ans = max_part; 
        pos = x;
    }
}
\end{lstlisting}

\subsubsection{树的直径}

\begin{lstlisting}
int dfs(int x) {
    int res1 = 0, res2 = 0;
    for (int i = 0; i < G[x].size(); ++i) {
        Edge& e = edges[G[x][i]];
        res2 = max(res2, dfs(e.to) + e.val);
        if (res2 > res1) swap(res1, res2);
    }
    ans = max(ans, res1 + res2);
    return res1;
}

结果:ans
\end{lstlisting}

\subsubsection{最近公共祖先(LCA)}

模板:$Luogu\ P3379$最近公共祖先

\begin{lstlisting}
int n, m, s;
int dep[maxn], f[maxn][20], lg[maxn];

void dfs(int u, int fa) {
    dep[u] = dep[fa] + 1;
    f[u][0] = fa;
    for (int i = 1; (1 << i) <= dep[u]; ++i) {
        f[u][i] = f[f[u][i - 1]][i - 1];
    }
    for (int i = 0; i < G[u].size(); ++i) {
        Edge& e = edges[G[u][i]];
        if (e.to != fa) {
            dfs(e.to, u);
        }
    }
}

int LCA(int x, int y) {
    if (dep[x] < dep[y]) swap(x, y);
    while (dep[x] > dep[y]) {
        x = f[x][lg[dep[x] - dep[y]] - 1];
    }
    if (x == y) return x;
    for (int k = lg[dep[x]] - 1; k >= 0; k--) {
        if (f[x][k] != f[y][k]) {
            x = f[x][k];
            y = f[y][k];
        }
    }
    return f[x][0];
}

lg数组的预处理:
for (int i = 1; i <= n; ++i) {
    lg[i] = lg[i - 1] + (1 << lg[i - 1] == i);
}
\end{lstlisting}

\section{字符串算法}

\subsection{KMP字符串匹配}

模板:$Luogu\ P3375$KMP字符串匹配

\begin{lstlisting}
char s1[maxn], s2[maxn]; int l1, l2, nxt[maxn];
void pre() {
    nxt[1] = 0;
    int j = 0;
    for (int i = 1; i < l2; ++i) {
        while (j > 0 && s2[j + 1] != s2[i + 1]) j = nxt[j];
        if (s2[j + 1] == s2[i + 1]) j++;
        nxt[i + 1] = j;
    }
}
void kmp() {
    int j = 0;
    for (int i = 0; i < l1; ++i) {
        while (j > 0 && s2[j + 1] != s1[i + 1]) j = nxt[j];
        if (s2[j + 1] == s1[i + 1]) j++;
        if (j == l2) {
            printf("%d\n", i - l2 + 2);
            j = nxt[j];
        }
    }
}
\end{lstlisting}

\subsection{字符串哈希}

模板:$Luogu\ P3370$字符串哈希

各种哈希:\url{https://tony031218.github.io/2019/01/10/Cpp算法-字符串算法-字符串哈希/}

\begin{lstlisting}
typedef unsigned long long ULL;
ULL base = 131, a[10010], mod = 19260817;
char s[10010];

ULL hash(char* s) {
    int len = strlen(s);
    ULL Ans = 0;
    for (int i = 0; i < len; i++)
        Ans = (Ans * base + (ULL)s[i]) % mod;
    return Ans;
}
\end{lstlisting}

\subsection{Trie字典树}

上面说过了


\section{实用技巧}

\subsection{别忘了文件输入}

\begin{lstlisting}
freopen("题目名.in", "r", stdin);
freopen("题目名.out", "w", stdout);
\end{lstlisting}

\textbf{文件名不要打错。。。}

\subsection{头文件}

在考场上,尽量不要使用\texttt{bits/stdc++.h}万能头,可能评测机没有这个库,但是我们可以去电脑里找这个头文件的位置,把其中要用的东西复制过来

\begin{lstlisting}
#include <cmath>
#include <cstdio>
#include <cstdlib>
#include <cstring>
#include <ctime>
#include <algorithm>
#include <bitset>
#include <deque>
#include <iostream>
#include <list>
#include <map>
#include <queue>
#include <set>
#include <stack>
#include <string>
#include <vector>
\end{lstlisting}

大概这些就是常用的头文件

\subsection{读入优化}

没什么好说的,大数据输入一定要用

\begin{lstlisting}
inline int read() {
    int x = 0; int f = 1; char ch = getchar();
    while (!isdigit(ch)) {if (ch == '-') f = -1; ch = getchar();}
    while (isdigit(ch))  {x = x * 10 + ch - 48; ch = getchar();}
    return x * f;
}

inline LL read() {
    int x = 0; int f = 1; char ch = getchar();
    while (!isdigit(ch)) {if (ch == '-') f = -1; ch = getchar();}
    while (isdigit(ch))  {x = x * 10 + ch - 48; ch = getchar();}
    return x * f;
}
\end{lstlisting}

建议在经常调用的函数前面加上\texttt{inline}

\subsection{随机数}

做不上的题目或者子任务,不要空下,按照输出格式输出随机数\\
记得在\texttt{main()}函数开头加上\texttt{srand((int)time(0));},再使用\texttt{rand()}

\subsection{时间检测}

因为考试环境是\texttt{Linux},我们可以使用本机命令\texttt{time}来测试自己的程序运行最大数据(题目提供)所需时间\\
命令:\texttt{time ./problem < problem.in > a.out}

\subsection{根据数据大小判断算法时间复杂度}

\noindent$n\leq 100$ \ -\  $O(n^3)$或更小\\
$n\leq 10000$ \ -\  $O(n^2)$或更小($O(n^2)$不一定)\\
$n\leq 100000$ \ -\  $O(n\cdot \sqrt{n})$或更小\\
$n\leq 1000000$ \ -\  $O(n\log n)$或更小\\
$n\geq 1000000$ \ -\ $O(n)$或$O(1)$ \\


大概就这些了,不够以后再补\\

{\Huge $NOIp\ 2019$ \texttt{rp++}}


\end{document}